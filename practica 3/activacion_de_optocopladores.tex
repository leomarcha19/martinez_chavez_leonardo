\documentclass[12pt,a4paper]{article}
\usepackage[utf8]{inputenc}
\usepackage{amsmath}
\usepackage{amsfonts}
\usepackage{amssymb}
\usepackage{graphicx}
\usepackage[left=2cm,right=2cm,top=2cm,bottom=2cm]{geometry}
\author{leonardo}
\title{Práctica 3: “activación de optocopladores”}
\begin{document}
\section{Objetivo: El Alumno aprenderá a controlar un foco por medio de transistores y optocopladores.}
\begin{flushleft}
\begin{itemize}
\item 1.	Computadora
2.	Software para simulación de circuitos electrónicos.
3.	Fuente de voltaje regulable.
4.	Diodos de acción rápida.
    5   Protoboard
    6   resistencia variable de 100k
7	optocopladores 4n25
    8   transistores n222A.
9   Multímetro
10  arduino uno
11   relay

\end{itemize}
\end{flushleft}
\section{desarrollo}
\begin{flushleft}
1.	Usando un simulador, arme el circuito de la Diagrama 1 sustituyendo la fuente de corriente por un medidor de corriente y de voltaje
2.	Conecte los optocopladores junto con las resistencias y los leds en serie cunto con el transistor n2222A como se explicó en los procedimientos anteriores y observe como se ven afectados los medidores de corriente y voltaje y grafique
3.	Arme el diagrama 2 usando los transistores de potencia para controlar el giro del foco dando 3 cantidades de iluminación diferentes.
\end{flushleft}
\section{conclusion}
en esta patica vimos como al conectarlo con el potenciomtro vemos como podemos controlar la intensidad del foco la cual nos costo trabajo para poder darles 3 diferentes tonos en el foco, por que el transistor en un momento alimentaba lo minimo pero para darle un tono mas bajo se apagaba el foco lo cual tuvimos que conectarle el potenciometro para poder darle el disparo y no se apague el foco de inmediato 

\end{document}
