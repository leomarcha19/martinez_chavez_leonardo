\documentclass[12pt,a4paper]{article}
\usepackage[utf8]{inputenc}
\usepackage{amsmath}
\usepackage{amsfonts}
\usepackage{amssymb}
\usepackage{graphicx}
\usepackage[left=2cm,right=2cm,top=2cm,bottom=2cm]{geometry}
\author{leonardo}
\title{Barra transportadora}
\begin{document}
\includegraphics[width=15cm]{1.jpg} 
\begin{flushleft}
Autor: Leonardo Martinez Chavez 
\end{flushleft}
\begin{flushleft}
Universidad Politécnica de la Zona Metropolitana de Guadalajara
\end{flushleft}
\begin{flushleft}
Ingeniería en Mecatronica                                              

\end{flushleft}
\begin{flushleft}
Noviembre 2019 
\end{flushleft}
\section{
La implementación de la barra transportadora para automatizar  los traslados de producto como método para maximizar recursos en la industria
}
\subsection{Planteamineto del problema}

\begin{flushleft}
En México no contamos con un dato  del número de lesiones músculo – esqueléticos  por exceso de carga en el área e la industria ya que en muchas acciones no se atienden  y si hablamos de las pérdidas económicas por un mal manejo de los productos tampoco se tiene claro la perdida que se tiene, ya que cada industria tiene costos de producción diferente.
Pero  la pérdida económica nunca es buena en ningún ámbito ya sea por incapacidades  de lesiones por cargar  y perdidas por daño de producto, los recursos que pueden ser invertidos en otras áreas de la  industria. 

\end{flushleft}
\subsection{Objetivos:}
\begin{flushleft}
objeivo general:
\end{flushleft}
\begin{center}
-	Realizar  un prototipo de sistema automatizado para el transporté de productos el cual podrá ser adaptado  tanto en productos del campo como productos procesados. 
\end{center}
\begin{flushleft}
Objetivos especificos:
\end{flushleft}
\begin{center}
-	Estudiar los diferentes tipos  de  bandas
\end{center}
\begin{center}
-	Analizar el  diseño que pueda ser mas manipulable 
\end{center}
\begin{center}
-	Describir la lista de materiales para  realizar el prototipo
\end{center}
\begin{center}
-	Diseñar el prototipo de la barra transportadora
\end{center}
 \subsection{Justificacion:}
 \begin{flushleft}
 La intención de realizar este proyecto es el que las empresas logre tener una mayor eficiencia en el traslado  de objetos con menos personal, esto les disminuiría el costo en incapacidades de lesiones por cargar, disminución del personal que se tendría que encargar del la movilización del producto, esto lograra un mejor manejo de los recursos económicos de las empresas.  
No solo se trata de las lesiones que puede tener el personal sino también de cuidar los artículos que se están movilizando, evitando que se pierda productos por un mal manejo. 
Es decir lograr mover más objetos en menor tiempo con menor costo. 


 \end{flushleft}
\subsection{Delimitacion}
\begin{flushleft}
en esta practica tuvimos muchos contratiempos en custeion de el tiempo, en el cual la mayoria trabajabamos, ademas por otras actividades en la materia entre otras cosas. 
\end{flushleft}
\begin{flushleft}
Otro contratiempo que tuvimos es que la barra gire  puesto que al inicio solo la el motor giaba sin la banda eso y que los circuito quedaran 
\end{flushleft}
\begin{flushleft}
\subsection{costos de las piezas}

4 baquelita ... 80.00
\end{flushleft}
\begin{flushleft}
tablas de madera 200.00
\end{flushleft}
\begin{flushleft}
malla metalica ... 30.00

\end{flushleft}
\begin{flushleft}
PLC... 1500.00

\end{flushleft}
\begin{flushleft}
cables duppons...80.00
\end{flushleft}
\begin{flushleft}
resistor diferentes medidas...15.00
\end{flushleft}
\begin{flushleft}
potenciometros diferentes medidas... 100.00
\end{flushleft}
\begin{flushleft}
motores... 200.00
\end{flushleft}
\begin{flushleft}
solenoides... 300.00
\end{flushleft}
\subsection{Aportacion de las materias}
\begin{flushleft}
programacion logica cableada:
\end{flushleft}
\begin{center}
con esta materia aprendimos a programar el plc el cual nos ayudo  para poder activar el motor y poder manejarlo mas ya sea con las entradas y salidas para poder activar el solenoide 
\end{center}
\begin{flushleft}
Ingles:
\end{flushleft}
\begin{center}
Esta materia nos ayudo para poder buscar mas informacion ya que la mayoria de automotizacion esta en ingles.
\end{center}
\begin{flushleft}
interfaz de potencia:
\end{flushleft}
\begin{center}
Esta materia nos ayudo para hacer el cableado y los circuitos, puesto que la mayoria del proyecto esta basado en las practicas de esa materia. Como segudo avance adicionamos unos sensores para poder darle mas uso al plc el cual nos ayudaria con entradas y salidas ademas de que ingresar el plc en el proyecto nos ayuda por si queremos ingresarle mas cosas el cdigo sea facil de cambiar  
\end{center}
\begin{flushleft}
Programacion de perifericos:
\end{flushleft}
\begin{center}
Por situaciones con la maestra esta materia no nos pudo ayudar en el proyecto
\end{center}
\subsection{Conclusion del proyecto:}
\begin{flushleft}
El hacer el proyecto fue mas laborioso de lo pensado ya que  hubo inprevistos o algunas cosas no fueron muy bien planeadas como que la banda tardo en gira ademas los sensores tambien el ingresarlos era algo dificil puesto que se desconectaban y eso causaba problemas ademas los solenoides no se puediron agregar como al final se tenia planeado el cual esto le quita puntos al trabajo o lo hace un trabajo no terminado.En cuestion de estetica nos falto pero eso la verdad no fue prioridad en el aspecto de que lo primero lo mas importante es que funcione en su totaidad en este caso una mejora no fue agregada pero no quiere decir de que el proyecto no sirviera.
\end{flushleft}
\begin{flushleft}
En fin  proyecto esta pensado para adicionarle materias del plan de carrera  y hacer que la barra sea mas eficiente hasta el final de la carrera 
\end{flushleft}
\subsection{Evidencia del proyecto}
\begin{flushleft}
\includegraphics[width=15cm]{1.jpeg} 
\end{flushleft}
\begin{flushleft}
\includegraphics[width=15cm]{2.jpeg} 
\end{flushleft}
\begin{flushleft}
\includegraphics[width=15cm]{3.jpeg} 
\end{flushleft}
\begin{flushleft}
\includegraphics[width=15cm]{4.jpeg} 
\end{flushleft}

\end{document}