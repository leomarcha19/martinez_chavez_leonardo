\documentclass[12pt,a4paper]{article}
\usepackage[utf8]{inputenc}

\usepackage{amsmath}
\usepackage{amsfonts}
\usepackage{amssymb}
\usepackage{graphicx}
\usepackage[left=2cm,right=2cm,top=2cm,bottom=2cm]{geometry}
\author{leonardo}
\begin{document}
\part{Universidad politecnica de la zona metropolitana de guadalajara }
\includegraphics[width=12cm]{upzmg.jpg}
  \begin{center}
  López Linares Barbara Monserrat.
  \end{center}
  \begin{center}
  Martínez Chávez Leonardo.
  \end{center}
  \begin{center}
  Ramos Sánchez David.
  \end{center}
  \begin{center}
  Tema: casa domotica
 \end{center}
\begin{center}
  4 cuartrimestre
  \end{center}  
  \begin{center}
  T/M
  \end{center}
  \section{Casa domótica}
  Nosotros decidimos hacer una casa domótica, puesto que estamos adaptando un lugar común para el bienestar de cada persona con la tecnología, teniendo en cuenta el cambio que está teniendo la sociedad en el cual la mayoría de las empresas o lugares que nos rodean están teniendo una implementación de tecnologías mejoradas.
Los usos de estas tecnologías son para la producción, fabricación y/o desempeño de las personas, las empresas ya están adaptando todo su entorno, por lo que decidimos hacer la mejora de una casa para cualquier tipo de persona, desde alguien que solo desee un lugar mejor o desde una discapacidad temporal o permanente.
Para hacer esta casa domótica hemos planeado usar nuestros conocimientos que se adquirirán basados en la materia de Programación de periféricos y Sistemas Electrónicos de Interfaz para adaptarla.
 Al igual por ser un Proyecto basado a tres cuatrimestres se tiene previsto la adquisición de mayores conocimientos para integrar más acciones a esta casa y que su funcionalidad sea la mejor posible para una persona específica, o dependiendo sus necesidades.
 \section{Planteamiento del problema }
 Lo que buscamos con nuestro proyecto es que el usuario pueda estar más cómodo y seguro en su vivienda teniendo la confianza para estar dentro o fuera de su hogar.
Actualmente la seguridad es un tema que se puede modernizar por así decirlo, con las instalaciones automáticas, el correr riesgos podrían evitarse. 
 Nosotros preferimos que estos riesgos se puedan evitar con una casa domótica la cual el cerrar la puerta del garaje no sea un problema, el salir y tener la confianza que la casa estará protegida. 
\section{Objetivo general del proyecto}
Lo que el equipo busca es diseñar un programa y un circuito con el cual podamos activar dispositivos de manera fácil para el bienestar propio y/o el de otras personas, se tiene planeado elaborar para un futuro una casa que pueda evitar actividades de riesgo dependiendo las necesidades del usuario, de las cuales están en proposición para en otra instancia ser agregadas a esta casa.
Se pretende que en esta casa domótica lo principal sea:
\begin{center}
•	  Abrir la puerta del garaje 
\end{center}
\begin{center}
•	Encender las luces
\end{center}
Se desea que la casa domótica ya concluida tenga integrado:\begin{center}
•	Detector de movimiento para encendido de luces 
\end{center}
\begin{center}
•Encendido de Ventilador al llegar a 30 grados
\end{center}
\begin{center}
•Una alarma de incendios 
\end{center}
\section{Justificación}
Principalmente la casa domótica está pensado en mejorar la calidad de vida de las personas además de avanzar tecnológicamente en la seguridad de la familia. Se pretende hacer este proyecto con el carácter informativo, puesto que esta casa podría ayudar a personas que sufren de algún tipo de enfermedad, parálisis o discapacidad, actualmente el ser humano está empeñado en avanzar tecnológicamente por lo que este recurso también podría ser implementado en otros lugares como hospitales o empresas que lo requieran
\begin{flushleft}
lista de materiales
\end{flushleft}
\begin{tabular}{|c|c|}
\hline 
• material & • precio \\ 
\hline 
arduino & 150 \\ 
\hline 
plc & 1500 \\ 
\hline 
casa de vidrio & 200 \\ 
\hline 
cables dupons & 20 \\ 
\hline 
leds & 10 \\ 
\hline 
protoboard & 80 \\ 
\hline 
raspberry pi & 1200 \\ 
\hline 
fotoresistencias  & 20 \\ 
\hline 
motoreductor & 100 \\ 
\hline 
\end{tabular} 
\begin{flushleft}
INGLÉS IV
\begin{center}
•Nos ayudara para la lectura de los documentos pdf  ya que la mayoría de información son en ingles.
\begin{flushleft}
ÉTICA PROFESIONAL
\end{flushleft}
\end{center}
\begin{center}
• La realización del trabajo en equipo.
\end{center}
\end{flushleft}
\begin{flushleft}
ESTRUCTURA Y PROPIEDADES DE LOS MATERIALES
\end{flushleft}
\begin{center}
•El decidir el material más conveniente para la casa, no solo dejándonos llevar por el costo del material si no por la resistencia que tendría para el soporte de lo que se utilizara. Además que le da más presentación y nos ayuda a ver mejor lo que pasa dentro de la casa.
\end{center}
\begin{flushleft}
PROGRAMACIÓN DE PERIFÉRICOS
\end{flushleft}
\begin{center}
• Para hacer el código de programación  para la o las tarjetas que serán requeridas adicional a eso podríamos apoyarnos con la interfaz.
\end{center}
\begin{flushleft}
SISTEMAS ELECTRÓNICOS DE INTERFAZ
\end{flushleft}
\begin{center}
•Ver que componentes podemos  utilizar, como se debe conectar  o alimentar parano tener ningún inconveniente en la casa, adicional a eso también nos va ayudar  con los componentes pasivos y activos .
\end{center}
\begin{flushleft}
CONTROLADORES LÓGICOS PROGRAMABLES
\end{flushleft}
\begin{center}
• Para el entendimiento de cómo podemos programar basándonos en el comportamiento de las compuertas y asi saber cuál será más adecuado utilizar.
\end{center}

\end{document}